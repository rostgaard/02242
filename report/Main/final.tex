\documentclass[11pt,a4paper]{article}
\newcommand{\confdir}{../conf/} % Set the folder which contains preamble.tex, titlepage.tex and the general img/ folder
\newcommand{\imgdir}{../img/} % Set the folder which contains the specific images for the report
\newcommand{\texdir}{../tex/} % Set the folder which contains the content (.tex files)
\input{\confdir preamble}
\usepackage{listings}
\usepackage[normalem]{ulem}
\usepackage[scaled]{beramono}
\usepackage{tikz}
\usetikzlibrary{shapes,arrows}
\usepackage{libertine}
\usepackage[T1]{fontenc}
\usepackage{color}
\definecolor{bluekeywords}{rgb}{0.13,0.13,1}
\definecolor{greencomments}{rgb}{0,0.5,0}
\definecolor{redstrings}{rgb}{0.9,0,0}
\usepackage{amsmath}
\usepackage{MnSymbol}
\usepackage{algpseudocode}
\usepackage{algorithm}
\definecolor{mygreen}{rgb}{0,0.6,0}
\definecolor{mygray}{rgb}{0.2,0.2,0.2}
\definecolor{mymauve}{rgb}{0.58,0,0.82}

\lstset{escapechar=\@}

% Define block styles
\tikzstyle{block} = [rectangle, draw, fill=white!20, 
    text width=3.2em, text centered, rounded corners, minimum height=3em]
\tikzstyle{line} = [draw, -latex']

%%%%%%%%%%%
%% SETUP %%
%%%%%%%%%%%

%% Set the course number and course name
% Usage: \setcourse{<course no>}{<course name>}
	\setcourse{02242}{Program Analysis}

%% Set the title of the report
% Usage: \settitle{<title>}
	\settitle{Exam Project}

%% Set the subtitle of the report
% Usage: \setsubtitle{<subtitle>}
	\setsubtitle{Program analysis tool}

%% Set the date the report is handed in
% Usage: \setdate{<hand-in date>}
	\setdate{December 2nd, 2013}

%% Add the authors of the report
% Usage: \addauthor{<study number>}{<first name(s)>}{<last name>}
	\addauthor{s093477}{Ibrahim}{Nemli}
	\addauthor{s084283}{Kim Rostgaard}{Christensen}
	\addauthor{s093263}{Peter Gammelgaard}{Poulsen}

	\setmoretext{}

%%%% END OF SETUP


\begin{document}

%% Insert titlepage
\inserttitlepage


\begin{abstract}
This is the report documenting and describing the work done on the project handed out in the course 02242: Program Analysis at the Technical University of Denmark. It documents the theory and design applied in a program analysis tool. The tool is intended to perform a minor set of analysis’s using a monotone framework and a worklist algorithm for solving equations and calculating a solution. The analysis' considered are reaching definition, sign detection and interval analysis. The results of these are used to calculate program slice and buffer overflow.
\end{abstract}

\clearpage

%% Insert table-of-contents
\inserttoc

%% Input files for the report
\import{\texdir}{requirements}

\bibliographystyle{plain}
\bibliography{references}

\end{document}
