% % \chapter{Implementation}
% % This section will first present the implementation of the parser for the \texttt{While} language. Then the implementations of the analysis' described in the previous section will be discussed.
% % \\\\
% % The Java language has been chosen as the implementation language.
% %  
% % \section{Parsing a program}
% % Before any analysis can take place it necessary that the program is parsed. This is done by having a parser. ANTLR is used as the framework for constructing a lexer and a parser that can take a program written in the extended \texttt{While} language and construct an abstract syntax tree (AST) using the data structures described in section~\ref{section:Abstractsyntaxtreedatastructure}. 
% 
% \section{Constructing the intermediate representation}
% As described in section~\ref{section:CreatingFlowGraphs} we use an unbalanced tree for containing our abstract syntax tree, which serves as our intermediate representation (IR).
% The implementation architecture for the IR is seen in Figure 1 - Figure 5. This architecture is mapped into java objects representing the program, each declaration, each statement etc.
% 
% \section{Function Mapping}
% In order to access information about each statement (block) we have implemented the smallest steps of each analysis to the individual classes (blocks) in the IR.
% \\\\
% For each \texttt{Statement} class we have implemented following functions:
% \begin{itemize}
% \item Label(s) - returns the set of label for the statement 		
% \item Init(s) - returns the initial label for the statement
% \item Final(s) - returns the set of final labels for the statement 
% \item Flow(s) - returns the flow for the statement
% \item Variable(s) - returns the set of variables for the statement
% \item Kill - returns the set of definitions to be killed	
% \item Gen - returns the set of definitions to be generated
% \end{itemize}
% 
% As an example for assignments and skip-statement we would have the following results when looking the statements \texttt{assignment} and \texttt{skip}.:
% \\
% \begin{tabular}{|l|l|l|}
% \hline
% \textbf{Statement}&		[x:=a]l&												[skip]l\\
% \hline
% \hline
% Label(s)&		\{l\}&													\{l\}	\\
% \hline
% Init(s)&			\{l\}&													\{l\}\\
% \hline
% Final(s)&		\{l\}&													\{l\}\\
% \hline
% Flow(s)&		Ø&													Ø\\
% \hline
% Variable(s)&		variable(a)&									Ø\\
% \hline
% Kill&			RDentry(l)\textbackslash\{x, l' | Bl' is an assignment to A\}&		Ø\\							
% \hline
% Gen&				\{x, l\}&												Ø\\
% \hline
% \end{tabular}
% 
% We have discusses most of the function above in the previous sections of the report. However the Variable function is new. The variable function is used to get variables for a particular statement. It returns a set of variables if any is represented.
% 
% If the statement is a declaration, we have defined the variable-function as following:\\
% 
% \begin{tabular}{|l|l|}
% \hline
% variable(decl)&:\\
% \hline\hline
% (Integer)&			x -> \{x\}\\
% \hline
% (Array)&				A[n] -> \{A\}\\
% \hline
% \end{tabular}
% 
% For expressions it is defined as:
% variable(expr):
% 	(constant) 			n -> Ø
% 	(variable)			x -> {x}
% 	(Array expr.)		A[x] -> {A} U var(x)
% 	(Arithmetic opr.)	a1 opa a2 -> var(a1) U var(a2)
% 	(Negation)			-a -> var(a)
% 	(Parenthesis)		(a) -> var(a)
% 
% And finally the statements:
% variable(stmt):
% 	(Assignment) 		x := a -> {x} U variable(x)	
% 	(Skip)				skip -> Ø
% 	(Array assignment)	A[a1] := a2 -> {A} U variable(a1) U variable(a2)
% 	(Read)				read x -> {x}
% 	(Read array)		read A[a] -> {A} U variable(a)
% 	(Write)				write a -> variable(a)
% 	(composition)		S1 S2 -> variable(S1) U variable(S2)
% 	(if-stmt)			if b then s1 else s2 fi -> Ø
% 	(while loop)		while b do S od -> Ø
% 	
% The variable-function is used for the Program Slicing algorithm.
% 	
% The functions explained above will give us the equations that is going to be solved.
% 
% \section{Implementation of the analysis}
% On top of the functions explained above we have implemented the algorithms which is able to perform the analysis as Reaching definitions, Program slicing, detection of signs and interval analysis. 
% Mode details of each analysis is explained below.
% 
% Implementation of the Reaching Definition analysis:
% The class WorklistAlgorithm.java implements the Reaching Definition analysis.
% It is an implementation of the pseudo-code presented in the slides for the course.
% 
% Implementation of the Program slice analysis:
% The class ProgramSlicing.java implements the program slice algorithm.
% It takes two parameters, a point of interest and the the result from reaching definition.
% It is an implementation of the pseudo-code presented in section 3.5 Program slice calculation algorithm.
% 
% ===========================================================================================================================

{\setlength{\chapterfontsize}{24pt}
\chapter{General about implementation}
}
This chapter will first present the implementation of the parser for the \texttt{While} language. Then the implementations of the analysis' described in the previous section will be discussed.
\\\\
The Java language has been chosen as the implementation language.
\\\\
For the implementation we have chosen to map the label $\ell$ to a Node object. The label is still present within the node and is used for representation but also contains additional information, such as the which statement it maps to.

\section{Parsing the program}
Before any analysis can take place it necessary that the program is parsed. This is done by having a parser. ANTLR is used as the framework for constructing a lexer and a parser that can take a program written in the extended \texttt{While} language and construct an abstract syntax tree (AST) using the data structures described in section~\ref{section:Abstractsyntaxtreedatastructure}. 
\begin{itemize}
	\item \textbf{The Lexer} - We've implemented the lexer as a grammar file for ANTLR, supplying us with a basic tokenizer that is able to attach extra information to the streamed tokens. The grammar file is located in the appended code for those interested.
	\item \textbf{The parser} - The parser is created by extending the grammar file hereby enabling it to expand the grammar to a generated parser. The parser harvests information from the lexer in order to provide an AST containing the information we need for out analysis` later on.
\end{itemize}


\section{Generalizing the analysis}
To abstract away the analysis from the general parser tool, we've created an interface that implemented by our abstract Statement class.
\begin{figure}
\centering
\begin{tikzpicture} 
\umlclass[type=interface]{Analyzable}{}{ 
%  + \textbf{\umlvirt{hasPotentialUnderFlow(Analysis[$\ell$]): Boolean}} \\
  + \umlvirt {labels () : NodeSet} \\
  + \umlvirt {initial () : Node} \\
  + \umlvirt {finalNodes () : NodeSet} \\
  + \umlvirt {flow () : FlowSet } \\
} 
\end{tikzpicture}
\caption{The Analyzable interface}
\label{fig:analysable_basic_definition}.
\end{figure}The interface basically maps the \textit{label(S)}, \textit{init(S)}, \textit{final(S)} and \textit{flow(S)} functions from table~\ref{table:flow_graph_functions} to abstract statement, so we can use it later on to harvest flows and labels via recursion. This is done using the interface \texttt{Analyzable} as defined in figure~\ref{fig:analysable_basic_definition}.
Specific objects has been created to hold all the definitions and statements. The class \texttt{StatementList} holds all the statements in the program. By simple asking for the \texttt{flow} of the \texttt{statementSet} the complete flow graph will be returned.
\\\\
In order to perform the analysis an abstract interface \texttt{Lattice} has been created. This defines the basic elements for performing an analysis such as $\iota$ and $\bot$. For each of the tree analysis' a class that implements this interface has been created.
\begin{itemize}
	\item \texttt{RDLattice} - Reaching definitions analysis.
	\item \texttt{SignsLattice} - Detection of signs analysis.
	\item \texttt{IntervalLattice} - Interval analysis.
\end{itemize}
If a new similar analysis needed to be implemented I would be easy to add a new class implementing this interface.

\subsection{Calculating a solution}
In order to solve the equations provided by the analysis' the general worklist algorithm for computing the MFP solution has been implemented. The algorithm is placed as the method \texttt{execute} in a \texttt{Program}. It takes a specific analysis as a parameter and returns a result of the analysis for each label in the program as a result.

%TODO Add more meat on this. Something about keeping track of the values internally, even when going out of bounds.
% The valid transitions -- and their states -- are depicted in figure \ref{figure:interval_states}, which illustrates the dead-end of $-\infty$ and $\infty$ quite efficient.
% 


\chapter{Improving precision}
\section{Deriving information from boolean expressions}
As the analysis is now, we have a crude over-approximation, and in order to improve that, we can harvest more information from the flow graph.\\\\
Up until now, we did not care about the conditions for loops or branches. For instance, given the code in figure \ref{fig:if_flow_example}, we \emph{could} derive information about the state of the variable y, depending on which branch we take. In essence, if we take the true branch, we know for certain that the value of y, is indeed the value of x - which is 5 (but that is less interesting in this case). If -- on the other hand -- we take the false branch, we know for certain that the value of y is \emph{every other value} than those of x.\\\\
\begin{figure}[h]
\centering
\begin{tikzpicture}[->,>=stealth',shorten >=1pt,auto,node distance=2.5cm,
                    semithick]
   \tikzstyle{block} = [rectangle, draw, fill=none, 
    text width=5em, text centered, rounded corners, minimum height=3em]
    
  \node[block] (1) {[x:=5]$^1$};
  \node[block] (2) [below of=2]  {if [x=y]$^2$};
  \node[block] (3) [below left of=2]  {[y:=x+2]$^3$};
  \node[block] (4) [below right of=2]  {[y:=x-2]$^4$};
  \node[block] (5) [below right of=3]  {[write~y]$^5$};

  
  \path (1) edge  node {} (2)
        (2) edge  node {} (3)
            edge  node {} (4)
        (3) edge  node {} (5)
        (4) edge  node {} (5);

\end{tikzpicture}
 \caption{}
 \label{fig:if_flow_example}
\end{figure} In order to raise this to a more general level, we follow the suggestion from \cite{02242_slides} and map two transfer functions to the statements corresponding to if and while, rather than one.

%TODO More stuff about this, and link it together with the paragraph above. (and stuff).

We will then use the functions defined table \ref{table:expression_mapping}. The basic idea, starting from $a_1 = a_2$, is that if we have two sets with definitions; then we can only enter the true branch if $\mathcal{A_S}[\![a_1]\!]$ and $\mathcal{A_S}[\![a_2]\!]$ have the same values. Thus, the only viable definitions that are allow to be passed along to the true branch, are the definitions that are both in $\mathcal{A_S}[\![a_1]\!]$ and $\mathcal{A_S}[\![a_2]\!]$. Hence, the intersection.\\\\
The $a_1 \neq a_2$ then becomes the complimentary set to the one from $a_1 = a_2$, the $a_1 < a_2$ %TODO complete this, and verify the table.

\subsection{Handling arithmetic expressions}
\begin{table}
\centering
\begin{tabular}{|r|l|}
\hline
Expression & Definition \\
\hline
$ a_1 = a_ 2$     & $ \mathcal{A_S}[\![a_1]\!] \bigcap \mathcal{A_S}[\![a_2]\!] $ \\
$ a_1 \neq a_ 2$  &  $ \left(\mathcal{A_S}[\![a_1]\!] \bigcap \mathcal{A_S}[\![a_2]\!]\right) \setminus \left(\mathcal{A_S}[\![a_1]\!]) \bigcup (B_{=} \setminus \mathcal{A_S}[\![a_2]\!]\right) $ \\
$ a_1 < a_ 2$     & $ \mathcal{A_S}[\![a_1]\!] \bigcup \left(\mathcal{A_S}[\![a_1]\!] \setminus \mathcal{A_S}[\![a_2]\!]\right) $ \\
$ a_1 > a_ 2$     & $ \mathcal{A_S}[\![a_2]\!] \bigcup \left(\mathcal{A_S}[\![a_2]\!] \setminus \mathcal{A_S}[\![a_1]\!]\right) $ \\
$ a_1 \leq a_ 2$  & $ \mathcal{A_S}[\![a_1]\!] \bigcup \left(\mathcal{A_S}[\![a_1]\!] \setminus \mathcal{A_S}[\![a_2]\!]\right) \bigcup \left(\mathcal{A_S}[\![a_1]\!] \bigcap \mathcal{A_S}[\![a_2]\!]\right) $ \\
$ a_1 \geq a_2$   & $ \mathcal{A_S}[\![a_2]\!] \bigcup \left(\mathcal{A_S}[\![a_2]\!] \setminus \mathcal{A_S}[\![a_1]\!]\right) \bigcup  \left(\mathcal{A_S}[\![a_1]\!] \bigcap \mathcal{A_S}[\![a_2]\!]\right)$ \\

%Beq  ::= A[a1] intersect A[a2]
%Bneq ::= (Beq \ A[a1]) U (Beq \ A[a2])
%Blt  ::= A[a1] U (A[a1] \ A[a2])
%Bgt  ::= A[a2] U (A[a2] \ A[a1])
%Bleq ::= Beq U Blt
%Blgt ::= Beq U Bgt

%        & \text{if } \mathcal{A}_S [\![a]\!]\widehat{\sigma} = \{-\}$ \\
\hline
\end{tabular}
\caption{Expression mapping}
\label{table:expression_mapping}
\end{table}


\section{Underflow detection}

For underflow detection we've stated by extending the Analyzable interface as illustrated in figure \ref{fig:analysable_underflow_extension}.
\begin{figure}
\centering
\begin{tikzpicture} 
\umlclass[type=interface]{Analysable}{}{ 
  + \textbf{\umlvirt{hasPotentialUnderFlow(Analysis[$\ell$]): Boolean}} \\
  + \umlvirt {labels () : NodeSet} \\
  $...$
} 
\end{tikzpicture}
\caption{Extension of the Analyseable interface}
\label{fig:analysable_underflow_extension}
\end{figure}

We then loop through each label in the set of labels of the program, and if the test fails, we push the problematic node to a NodeSet which then can be used to inform the callee (probably Main procedure) of places where problems could arise.

\section{Overflow detection optimizations}
The interval analysis we are performing for while-loops only takes account that the body is executed once, hence the result of the interval analysis performed in the body is an under approximation, since there is potential for that the body will be executed multiple times. However the obtained results are indeed a part of the precise result, but on the other hand we might omit some of the results for interval analysis which might result an unprecise result for detection of buffer overflow.
If we could guarantee a number of iterations that the body would execute, we could do an optimization such as we could get a more exact result for the interval analysis by performing the analysis as the same number of iterations for the statements in the body.   
A reason for why we can not calculate the number of iterations for how many times the loop is executed is because it might be depended on the body, and it would require advanced analysis to calculate this number.
Another approach for how the interval analysis could be performed for while-loop is to a assume that the body will run infinity often. However, during the analysis the statements of the body has to know that they are in the context of a while-loop. The result obtain form this analysis will be an over approximation.

%\begin{figure}
%\centering
%\begin{tikzpicture} 
%\umlclass[type=abstract]{Expression}{}{ 
%  + \umlvirt{isOutOfBounds(Analysis[$\ell$]) : Boolean} \\
%  + \textbf{\umlvirt{hasPotentialUnderFlow(Analysis[$\ell$]): Boolean}} \\
%  + \umlvirt{evalulate(Analysis[$\ell$])} : SignSet\\
%  + \umlvirt{evalulate(Analysis[$\ell$]) : Interval}
%} 
%\end{tikzpicture}
%\caption{Extension of the Expression class}
%\end{figure}

%\begin{figure}
%\centering
%\begin{tikzpicture} 
%\umlclass[type=abstract]{Statement}{}{ 
%  + \umlvirt{isOutOfBounds(Analysis[$\ell$]) : Boolean} \\
%  + \textbf{\umlvirt{hasPotentialUnderFlow(Analysis[$\ell$]): Boolean}}
%} 
%\end{tikzpicture}
%\caption{Extension of the Statement class}
%\end{figure}


\section{Functions for underflow detection}
\begin{table}[h]
\begin{tabular}{| l | l |}
  \hline
  Statement & Function \\
  \hline
  \hline
  [A[a$_1$] := a$_2$]$^\ell$ & f$_\ell^{UF} (\widehat{\sigma}) = 
     \begin{cases} 
        \widehat{\sigma}[true   & \text{if } \{-\} \in \mathcal{A}_S [\![\text{a}]\!]\widehat{\sigma}\\
        \widehat{\sigma}[false  & \text{otherwise} \\
     \end{cases}$\\
  \hline
  [read A[a]]$^\ell$ & f$_\ell^{UF} (\widehat{\sigma}) = 
     \begin{cases} 
        \widehat{\sigma}[true   & \text{if } \{-\} \in  \mathcal{A}_S [\![\text{a}_1]\!]\widehat{\sigma}\text{ } \vee \text{ f}_\ell^{UF} (\text{a}_2,\widehat{\sigma}) \\
        \widehat{\sigma}[false  & \text{otherwise} \\
     \end{cases}$\\
  \hline
  [write A[n]]$^\ell$ & f$_\ell^S (\widehat{\sigma}) = \widehat{\sigma}$\\
  \hline
  [write a]$^\ell$ & f$_\ell^{UF} (\widehat{\sigma})$ = f$_\ell^{UF} (a, \widehat{\sigma})$\\
  \hline
  [b]$^\ell$ & f$_\ell^{UF} (\widehat{\sigma})$ = f$_\ell^{UF} (b, \widehat{\sigma})$\\
  \hline
  [skip]$^\ell$ & f$_\ell^{UF} (\widehat{\sigma})$ = false\\
  \hline
  [read x]$^\ell$ & f$_\ell^{UF} (\widehat{\sigma})$ = false\\
  \hline
  [x := a]$^\ell$ & f$_\ell^{UF} (\widehat{\sigma})$ = false\\
  \hline
\end{tabular}
\centering
\caption{Underflow functions for statements}
\label{table:underflow_functions_statements}
\end{table}


\begin{table}[h]
\begin{tabular}{| l | l |}
  \hline
  Expression & Motivation\\
  \hline
  \hline
  $\text{f}_\ell^{UF} (\text{x},\widehat{\sigma}) = false $ & No underflow can happen in \\ 
                                                            & identifiers for variables.\\
  \hline
  $\text{f}_\ell^{UF} (\text{n},\widehat{\sigma}) = false $ & Numerals don't underflow.\\
  \hline
  $\text{f}_\ell^{UF} ((\text{a}),\widehat{\sigma}) = \text{f}_\ell^{UF} (\text{a},\widehat{\sigma}) $ & Parentheses expression merely forwards.\\
  \hline
  $\text{f}_\ell^{UF} (\text{a}_1 \text{ op}_R \text{ a}_2,\widehat{\sigma}) = \text{f}_\ell^{UF} (\text{a}_1,\widehat{\sigma}) \vee \text{f}_\ell^{UF} (\text{a}_2,\widehat{\sigma}) $ & If either expression has an overflow,\\
                                & the entire expression overflows.\\
  \hline
  $\text{f}_\ell^{UF} (-\text{a},\widehat{\sigma}) = \text{f}_\ell^{UF} (\text{a},\widehat{\sigma}) $ & Underflow is unaffected by negation.\\
  \hline
  $\text{f}_\ell^{UF} (\text{A}[\text{a}],\widehat{\sigma}) = 
     \begin{cases} 
        \widehat{\sigma}[true   & \text{if } \{-\} \in \mathcal{A}_S [\![\text{a}]\!]\widehat{\sigma}\\
        \widehat{\sigma}[false  & \text{otherwise} \\
     \end{cases}
   $ & An index is defined as being non-negative.\\
  \hline
\end{tabular}
\centering
\caption{Underflow functions for expressions}
\label{table:underflow_functions_expressions}
\end{table}


\section{Odds and ends}
This section is home to the miscellaneous ideas, thoughts, implementations and ramblings that didn't quite fit any other sections.

\subsection{Undefined symbols}
While strictly not within the scope of the project, we found that odd \texttt{NullPointerExceptions} occurred when we tried to access symbols that had not been previously defined in the program. While this job would be best left to the parser\footnote{Which would be rather easy, as we have an elegant ``declarations before statements'' language.}, the pending deadline suggested no to fix what wasn't broken and instead implement a lookup that throws an \texttt{UndefinedVariableException} if the symbol is not known within our analysis'.

%TODO Write about strategies for handling arrays; treating them as a single variable versus expanding the array to n variables - where n is the size.
%TODO Something about the worklist and iota.


\todo{Udvid implementations afsnit}