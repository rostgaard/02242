\section{Program Slicing}
A program slicing is a set of statements that may affect the values at some point of interest.

\subsection{Example program}\label{sec:exampleprogram}
Figure~\ref{fig:codeexample} shows an example program written in the \texttt{WHILE} language. This example program will be used as an example for this exercise. The program consists of 3 declared variables which are assigned several times and a while loop.
\begin{figure}
	\begin{lstlisting}
	program
	[int x]@$^1$@
	[int y]@$^2$@
	[int z]@$^3$@
	[y := x]@$^4$@
	[z := 1]@$^5$@
	while [y>0]@$^6$@ do
	   [z:= z*y]@$^7$@
	   [y:= y-1]@$^8$@
	od
	[y:=0]@$^9$@
	end
	\end{lstlisting}
	\caption{Code used in to calculate program slice example.}
	\label{fig:codeexample}
\end{figure}
\\\\
For this program it is possible to calculate the program slice for a point of interest.
\begin{itemize}
	\item If the point of interest is label 8. The result of program slice analysis would be; [y:=x]$^4$, [y:=y-1]$^8$.
	\item If instead the point of interest is label 7.  The result of program slice analysis would be; [y:=x]$^4$, [z:=1]$^5$, [z:=z*y]$^7$, [y := y-1]$^8$.
\end{itemize}

\subsection{Reaching Definitions Analysis}
Our algorithm for calculating the Program Slice will use the result of a Reaching Definition Analysis. A reaching definition analysis determines statically which definitions may reach a point of interests.
\\\\
Table~\ref{table:rd_equations} shows the Reaching Definitions Analysis table for the extended \texttt{WHILE} language. For each statement it is possible to see what is generated and what is killed. We do and over-approximation with regards to arrays, in the sense that we do not care about the individual positions in the array, but the values of \emph{any} position of the array. In this context, an array inherits some of the properties of an assignment, but as we cannot guarantee anything about which index was changed, we must clear all previous definitions.
\begin{table}[h]
    \begin{tabular}{l | l }
    \textbf{A[a$_1$] = a$_2$} & kill$_{RD}$([A[a$_1$]]$^\ell$) = $\emptyset$ \\
                              & gen$_{RD}$([A[n]$^l$) = $\{(A[0],l), ... (A[i-a],l)\}$ \\

    \hline
    \textbf{read x} & kill$_{RD}$(read x) = \{ (x,l' \}|$B^{l'}$ is a declaration or an assignment to x \} \\
                              & gen$_{RD}$(read x) = $\{(x,l)\}$ \\
							  
    \hline
    \textbf{read A[a]} & kill$_{RD}$(read A[a]) = \{ (A[a],l' \}|$B^{l'}$ is a declaration or an assignment to A[a] \} \\
                              & gen$_{RD}$(read A[a]) = $\{(A[a],l)\}$ \\
							  
    \hline
    \textbf{write a} &  kill$_{RD}$([write a]$^l$) = $\emptyset$ \\
                   &  gen$_{RD}$([[write a]$^l$) = $\emptyset$ \\

    \hline
    \textbf{x:=a} & kill$_{RD}$(x:=a) = \{ (x,l' \}|$B^{l'}$ is a declaration or an assignment to x \} \\
                              & gen$_{RD}$(x:=a) = $\{(x,l)\}$ \\
    \end{tabular}
    \centering
	\caption{RD Equations}
	\label{table:rd_equations}
\end{table}
\\\\
Furthermore table~\ref{table:rd_analysis} shows the Reaching Definitions Analysis. By using the analysis and equations it is possible for a specific program to calculate the reaching definitions.

\begin{table}
\begin{tabular}{ l | l }
  \hline
  \textbf{$RD_{entry}(l)$} & $\begin{cases} \emptyset & \text{ if } \ell=init(S_*) \\ 
                                                                 \cup \{RD_{exit}(\ell')|(\ell',\ell)\in flow(S_*)\}  & \text{ otherwise }
                                                   \end{cases}$\\\\
  \hline
      
    \textbf{$RD_{exit}(l)$} & $(RD_{entry}(l)\backslash kill_{RD}(B^l))\cup gen_{RB}(B^l)$\\
    					& where $B_l\in blocks(S_*)$\\

  \hline
\end{tabular}
\centering
\caption{Analysis definitions}
\label{table:rd_analysis}
\end{table}


\subsection{Flow graph for program}
A flow graph can be generated from a program in the \texttt{WHILE} language. Table~\ref{table:flow_graph_definitions} shows what a flow graph consists of. In table~\ref{table:example_flow_table} our example program is shown using this definition.
\begin{table}
    \begin{tabular}{l | l }
    label$(S)$ & The set of nodes of the flow graph $S$ \\
    \hline    
    Init$(S)$  & The initial node of the flow graph $S$.\\
               & Unique node where the execution of the program starts.\\
    \hline
    Final$(S)$   & The final node of the flow graph $S$.\\
                 & A set of nodes where the execution of the program may terminate.\\
    \hline
    Block$(S)$   & A set of the blocks/statements in the program under inquisition.\\
    \hline
    Flow$(S)$& The edges of the flow graph for $S$. A set of pairs is returned. \\
    \end{tabular}
    \centering
	\caption{Function definitions}
	\label{table:flow_graph_definitions}
\end{table}
\begin{table}
    \begin{tabular}{l | l }
     Label($S$)   & $\{   1,2,3,4,5,6,7,8,9   \}$ \\
     \hline
     Initial($S$) & $1$ \\
     \hline
     Final($S$)   & $\left\lbrace   9   \right\rbrace$ \\
     \hline
     Blocks($S$)  & $\{$int x, int y, int z y:=x, z:=1, y>0, z:=$z\cdot y$, y:=y-1, y:=0 $\}$ \\
     \hline
     Flow($S$)    &  $\{ (1,2), (2,3), (3,4), (4,5), (5,6), (6,7), (7,8) (8,6), (6,9) \}$ \\
    \end{tabular}
    \centering
	\caption{Example flow graph table}
	\label{table:example_flow_table}
\end{table}
\\\\
Furthermore a graphical representation can be created of a flow graph to give a better overview. Figure~\ref{fig:flowgraph2} shows the Flow Graph for our example program.
\begin{figure}[h]
\centering
\begin{tikzpicture}
  [scale=0.8,auto=left,every node/.style={circle,fill=blue!20}]
  \tikzstyle{line} = [draw, -latex']

  \node[block] (1) at (0,3)  {[int x]$^1$};
  \node[block] (2) at (0,1)  {[int y]$^2$};
  \node[block] (3) at (0,-1)  {[int z]$^3$};
  \node[block] (4) at (0,-3)  {[y:=x]$^4$};
  \node[block] (5) at (0,-5)  {[z:=1]$^5$};
  \node[block] (6) at (0,-7)  {[y>0]$^6$};
  \node[block] (7) at (0,-9)  {[z:=z$\cdot$y]$^7$};
  \node[block] (8) at (0,-11)  {[y:=y-1]$^8$};
  \node[block] (9) at (3,-7)  {[y:=0]$^9$};

  
  \path [line] (0,4) -- (1);
  \path [line] (1) -- (2); %TODO; Add label here.
  \path [line] (2) -- (3);
  \path [line] (3) -- (4);
  \path [line] (4) -- (5);
  \path [line] (5) -- (6);
  \path [line] (6) -- (7);
  \path [line] (7) -- (8);
  \path [line] (8) -| (-2,-8) |- (6);
  \path [line] (6) -- (9);

\end{tikzpicture}
 \caption{Flow graph}

 \label{fig:flowgraph2}
\end{figure}

\subsection{Data flow equations}
Using the analysis definitions from table~\ref{table:rd_analysis} the following equations in table~\ref{table:equations} can be made for our example program.
\begin{table}
	\begin{tabular}{| l |}
		\hline
$RD_\circ(1)=\emptyset$\\
$RD_\circ(2)=RD_\bullet(1)$\\
$RD_\circ(3)=RD_\bullet(2)$\\
$RD_\circ(4)=RD_\bullet(3)$\\
$RD_\circ(5)=RD_\bullet(4)$\\
$RD_\circ(6)=RD_\bullet(1)\cup RD_\bullet(8)$\\
$RD_\circ(7)=RD_\bullet(6)$\\
$RD_\circ(8)=RD_\bullet(7)$\\
$RD_\circ(9)=RD_\bullet(6)$\\
\hline
\hline
$RD_\bullet(1)=(RD_\circ(1)\backslash\{(x,1)\})\cup \{(x,1)\}$\\
$RD_\bullet(2)=(RD_\circ(2)\backslash\{(y,2),(y,4),(y,8),(y,9)\})\cup \{(y,2)\}$\\
$RD_\bullet(3)=(RD_\circ(3)\backslash\{(z,3),(z,5),(z,7)\})\cup \{(z,3)\}$\\
$RD_\bullet(4)=(RD_\circ(4)\backslash\{(y,2),(y,4),(y,8),(y,9)\})\cup \{(y,4)\}$\\
$RD_\bullet(5)=(RD_\circ(5)\backslash\{(z,3),(z,5),(z,7)\})\cup \{(z,5)\}$\\
$RD_\bullet(6)=RD_\circ(6)$\\
$RD_\bullet(7)=(RD_\circ(7)\backslash\{(z,3),(z,5),(z,7)\})\cup \{(z,7)\}$\\
$RD_\bullet(8)=(RD_\circ(8)\backslash\{(y,2),(y,4),(y,8),(y,9)\})\cup \{(y,8)\}$\\
$RD_\bullet(9)=(RD_\circ(9)\backslash\{(y,2),(y,4),(y,8),(y,9)\})\cup \{(y,9)\}$\\
\hline
	\end{tabular}
    \centering
    \caption{Equations}
    \label{table:equations}
\end{table}
\\\\
These equations can be solved using the equations from~\ref{table:rd_equations}. The solutions can be found in table~\ref{table:solution1} and table~\ref{table:solution2}.
\begin{table}
	\begin{tabular}{| l | l |}
	  \hline
	  l & $RD_\circ(l)$ \\
	  \hline
	  \hline
	1 & $\emptyset$\\
	2 & $\{(x,1)\}$\\
	3 & $\{(x,1),(y,2)\}$\\
	4 & $\{(x,1),(y,2),(z,3)\}$\\
	5 & $\{(x,1),(y,4),(z,3)\}$\\
	6 & $\{(x,1),(y,4),(z,5),(z,7),(y,8)\}$\\
	7 & $\{(x,1),(y,4),(z,5),(z,7),(y,8)\}$\\
	8 & $\{(x,1),(y,4),(z,7),(y,8)\}$\\
	9 & $\{(x,1),(y,4),(z,5),(z,7),(y,8)\}$\\
	  \hline
	\end{tabular}
    \centering
    \caption{Solution to equation}
    \label{table:solution1}
\end{table}

\begin{table}
	\begin{tabular}{| l | l |}
	  \hline
	  l & $RD_\bullet(l)$ \\
	  \hline
	  \hline
	1 & $\{(x,1)\}$\\
	2 & $\{(x,1),(y,2)\}$\\
	3 & $\{(x,1),(y,2),(z,3)\}$\\
	4 & $\{(x,1),(y,2),(z,3)\}$\\
	5 & $\{(x,1),(y,4),(z,5)\}$\\
	6 & $\{(x,1),(y,4),(z,5),(z,7),(y,8)\}$\\
	7 & $\{(x,1),(y,4),(z,7),(y,8)\}$\\
	8 & $\{(x,1),(z,7),(y,8)\}$\\
	9 & $\{(x,1),(z,5),(z,7),(y,9)\}$\\
	  \hline
	\end{tabular}
    \centering
	\caption{Solution to equation}
	\label{table:solution2}
\end{table}

\subsection{Program slice calculation algorithm}
The Reaching Definitions Analysis algorithm can be used to calculate the Program Slice for a given program at a point of interest. An algorithm for calculating the Program Slice is given in figure~\ref{fig:programslicealgorithm}.
\\
\\
The algorithm takes the label of the point of interest as input. The output is an array with the labels that is a part of the program slice. It is assumed that the Reaching Definitions Analysis has been performed beforehand and the result is available in the variable \texttt{RD}. The algorithm uses a queue where the point of interest is added to and continues until the queue is empty. For every iteration the label in the queue is added to the \texttt{result} array. Then all the variables that this label depends on is retrieved. The labels where these variables are used from the result of the Reaching Definitions Analysis is added to the queue. It is assumed that once a label has been added to the queue it can not be added again. This will then guarantee that the algorithm terminates.
\begin{algorithm}
 \begin{algorithmic}[1]
 \Procedure{Program Slice}{$point\_of\_interest$}
 \State Array $result$
 \State Queue $queue$
 \State $queue.enqueue(point\_of\_interest)$
 \While{$queue.not\_empty()$}
 \State Label $l:=queue.dequeue()$
 \State $result.add(l)$
 \State Array $variables:=l.get\_variables()$
 \State Array RD = $RD_{entry}(l)$
\For {label in RD}
\If {$variables.contains(label.get\_variables())$}
\State queue.enqueue(label)
\EndIf 
\EndFor
 \EndWhile
 \State \textbf{return} $result$
 \EndProcedure
 \end{algorithmic}
 \caption{Calculate Program Slice}
 \label{fig:programslicealgorithm}
\end{algorithm}
\\\\
The algorithm can calculate the Program Slice for the example program from section~\ref{sec:exampleprogram}. We have the solution to the reaching definition analysis from table~\ref{table:solution1} available. To find the Program Slice with point of interest at label 8.
\begin{itemize}
	\item First the point of interest is added to the queue. $queue=[8]$, $result=[]$
	\item The condition is checked. The first element from the queue is removed and added to the result. $queue=[]$, $result=[8]$
	\item The variables from label 8 is retrieved. This is only $y$.
	\item The result of the reaching definition analysis for label 8 is retrieved. This is $\{(x, 1), (y, 4), (z, 7), (y, 8)\}$.
	\item Each of these labels are iterated. If any of them contains the variable $y$ they will be added to the queue. This is 4 and 8. Since is it assumed that a label can not be added to a queue more than once only 4 will be added. $queue=[4]$, $result=[8]$.
	\item The condition is checked again. The first element from the queue is removed and added to the result. $queue=[]$, $result=[8,4]$
	\item The variables from label 4 is retrieved. This is $x$.
	\item The result of the reaching definition analysis for label 4 is retrieved. This is $\{(x, 1), (y, 2), (z, 3)\}$.
	\item Each of these labels are iterated. If any of them contains the variable $x$ they will be added to the queue. This is 1. $queue=[]$, $result=[1]$.
	\item The condition is checked again. The first element from the queue is removed and added to the result. $queue=[]$, $result=[8,4,1]$
	\item The variables from label 4 is retrieved. There is none.
	\item The condition is checked again. It evaluates to false and the result is returned. $result=[8,4,1]$
\end{itemize}
The reached result is the same as the manual solution from section~\ref{sec:exampleprogram}.