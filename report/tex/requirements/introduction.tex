\section{Introduction}
Program analysis is the discipline of extracting and deriving information about the structure, and potential behavior of computer programs. It has many applications, such as; bug finding, optimizations, code reuse, formal validation and model mapping. In this report, we take a closer look at some of the specialized and general theory used for transforming source code text into analysis-friendly structures, and then performing the actual analysis on them. This is done by parsing a program into an abstract syntax tree, before mapping it into a flow graph that can be used together with a worklist algorithm to solve the equations from an analysis.
\\
\\
The next section will cover the choices of data structures for the abstract syntax tree and the flow graph. Then the analysis reaching definitions will be defined and it will be presented how to use it to calculate a program slice. Next detection of signs analysis will be used to find buffer overflows and followed by how interval analysis can be used to detect buffer overflows as well. The next section will present how these analysis' have been implemented in a java application. Finally the conclusion will sum up the important findings.